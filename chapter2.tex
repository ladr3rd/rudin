\section{Basic Topology}

\begin{questions}
  \question Prove that the empty set is a subset of every set.
  \begin{solution}
    Let $A$ be a set. The implication $x\in\emptyset\implies x\in A$ holds because $x\in\emptyset$ is never true, hence $\emptyset\subset A$.
  \end{solution}

  \question A complex number $z$ is said to be \emph{algebraic} if there are integers $a_0,\ldots,a_n$, not all zero, such that
  \[ a_0z^n + a_1z^{n-1} + \cdots + a_{n-1}z + a_n = 0. \]
  Prove that the set of all algebraic numbers is countable. \emph{Hint:} For every positive integer $N$ there are only finitely many equations with
  \[ n + \abs{a_0} + \abs{a_1} + \cdots \abs{a_n} = N. \]
  \begin{solution}
    Let $A_N$ denote the set of all complex numbers $z$ such that there exist integers $a_0,\ldots,a_n$ not all zero, such that $n+\abs{a_0}+\abs{a_1}+\cdots+\abs{a_n}=N$ and
    \[ a_0z^n + a_1z^{n-1} + \cdots + a_{n-1}z + a_n = 0. \]
    Then for each $N\in\N$, $A_N$ is finite, as from the hint there are only finitely many equations with $n+\abs{a_0}+\abs{a_1}+\cdots+\abs{a_n}=N$, and each of these equations has finitely many solutions.

    It is clear that $\bigcup_{N\in\N} A_N$ is the set of all algebraic numbers. Being a countable union of at most countable sets, it is at most countable. Every natural number $n$ is an algebraic number, being a solution to $z-n=0$, so the set of all algebraic numbers is infinite and hence countable.
  \end{solution}

  \question Prove that there exist real numbers which are not algebraic.
  \begin{solution}
    The set of real numbers is uncountable. The set of complex algebraic numbers is countable by the previous exercise, hence the set of real algebraic numbers is at most countable. Hence the set of real numbers cannot equal the set of real algebraic numbers, so there exist real numbers which are not algebraic.
  \end{solution}

  \question Is the set of all irrational real numbers countable?
  \begin{solution}
    If the set $\R\setminus\Q$ were countable, then as $\Q$ is countable, $\R=\Q\cup(\R\setminus\Q)$ would be an at most countable union of countable sets, hence $\R$ would be countable. We know that $\R$ is uncountable, so $\R\setminus\Q$ must also be uncountable.
  \end{solution}

  \question Construct a bounded set of real numbers with exactly three limit points.
  \begin{solution}
    Let
    $$S=\{1/n\,|\,n\in\N\}\cup\{1+1/n\,|\,n\in\N\}\cup\{2+1/n\,|\,n\in\N\}.$$
    We see immeditaely that $S$ is bounded as $[0,3]=N_{3/2}(3/2)\supset S$. Let $p\in\R$ be a limit point of S. If $p\notin[0,3]$, set $\varepsilon=\min\{d(p,0), d(p,3)\}>0$. Then clearly $N_{\varepsilon/2}(p)$ contains no elements of $S$, a contradiction. Furthermore, observe that $N_{1/2}(3)\bigcap S=\{3\}$, so 3 is not a limit point of $S$. Hence $p\in[0,3)$.
    
    Suppose there is $n\in\{0,1,2\}$ such that $p\in(n,n+1)$. Clearly this $n$ is unique. Then there is a minimal $k\in\N$ such that $n+1/k<p$, and $k>1$. Then $p\in[n+1/k, n+1/(k-1)]$, however $[n+1/k, n+1/(k-1)]\bigcap S = \{n+1/k, n+1/(k-1)\}$. If $p$ is in the open set $(n+1/k, n+1/(k-1))$, there is a neighborhood of $p$ that lies in $(n+1/k, n+1/(k-1))$ and hence contains no elements of S other than $p$, which is impossible. So $p=n+1/\ell$ for some integer $\ell>1$ (since it was assumed $p\notin\Z$). So let
    $$\varepsilon=\frac{1}{2}\left(p-\left(n+\frac{1}{\ell+1}\right)\right)=\frac{1}{2\ell(\ell+1)}$$
    Then $N_\varepsilon(p)$ contains no points of $S$ other than $p$. Hence $p$ is not a limit point of $S$ in this case either. Suppose, then, that $p$ is any element of the set
    $$[0,3)\setminus((0,1)\cup(1,2)\cup(2,3))=\{0,1,2\}.$$
   and let $r>0$. There exists $n\in\N$ such that $1/n<r$, hence $S\ni p+1/n\in N_r(p)$, so that the limit points of $S$ are exactly the points 0, 1 and 2.
  \end{solution}


  \question Let $E'$ be the set of all limit points of a set $E$. Prove that $E'$ is closed. Prove that $E$ and $\cl{E}$ have the same limit points. (Recall that $\cl{E}=E\cup E'$.) Do $E$ and $E'$ always have the same limit points?
  \begin{solution}
    Let $x$ be a limit point of $\cl{E}$, and let $r>0$. Then $N_r(x)$ contains a point $y\neq x$ such that $y\in\cl{E}$. If $y\notin E$ then $y$ is a limit point of $E$. Let $\varepsilon=\min\{r-d(y,x),d(y,x)\}>0$, then $N_\varepsilon(y)$ contains a point $z\in E$. Now
    \[ d(z,x) \leq d(z,y) + d(y,x) < \varepsilon + d(y,x) \leq r,  \]
    hence $z\in N_r(x)$. Moreover $d(z,x)\geq d(y,x)-d(y,z) > d(y,x)-\varepsilon \geq 0$, so $z\neq x$. So there exists a point of $E$ in $N_r(x)$ distinct from $x$, for all $r>0$. So all limit points of $\cl{E}$ are limit points of $E$. Clearly a limit point of $E$ is a limit point of $\cl{E}=E\cup E'$, so $E$ and $\cl{E}$ have the same limit points.

    Let $x$ be a limit point of $E'$, then $x$ is a limit point of $\cl{E}$, and hence a limit point of $E$. Therefore $x\in E'$, which shows that $E'$ is closed.

    Since $E'$ is closed, all limit points of $E'$ are limit points of $E$. The converse however is not true. Let $E$ be the set of all $1/n$, where $n\in\N$. Then $E'=\{0\}$, and so there are no limit points of $E'$. So $E$ and $E'$ do not always have the same limit points.
  \end{solution}

  \question Let $A_1,A_2,A_3,\ldots$ be subsets of a metric space.
  \begin{parts}
    \part If $B_n=\bigcup_{i=1}^n A_i$, prove that $\cl{B_n}=\bigcup_{i=1}^n\cl{A_i}$, for $n=1,2,3,$ \ldots.
    \begin{solution}
      Let $p\in\cl{B_n}$. If $p\in B_n$ then $p\in A_i$ for some $1\leq i\leq n$, hence $p\in \cl{A_i}$, and so $p\in\bigcup_{i=1}^n \cl{A_i}$. If $p\notin B_n$ then $p$ is a limit point of $B_n$.

      Suppose that $p$ is not a limit point of any of $A_1,A_2,\ldots,A_n$. Then there exist $\varepsilon_1,\varepsilon_2,\ldots,\varepsilon_n>0$ such that $N_{\varepsilon_i}(p)$ contains no points of $A_i$ for all $1\leq i\leq n$. Let $\varepsilon=\min\{\varepsilon_1,\varepsilon_2,\ldots,\varepsilon_n\}>0$, then $N_\varepsilon(p)$ contains no points of $\bigcup_{i=1}^n A_i=B_n$, which contradicts $p$ being a limit point of $B_n$.

      Therefore $p$ is a limit point of at least one $A_i$ with $1\leq i\leq n$, hence $p\in A_i'\subset\cl{A_i}\subset\bigcup_{i=1}^n\cl{A_i}$. It follows that $\cl{B_n}\subset\bigcup_{i=1}^n\cl{A_i}$. Conversely, let $p\in\cl{A_i}$ for any $1\leq i\leq n$. If $p\in A_i$, then $p\in B_n\subset\cl{B_n}$. Otherwise $p$ is a limit point of $A_i$, so it is a limit point of $B_n$, hence $p\in\cl{B_n}$. It follows that $\bigcup_{i=1}^n \cl{A_i}\subset\cl{B_n}$, which proves the claim.
    \end{solution}

    \part If $B=\bigcup_{i=1}^\infty A_i$, prove that $\cl{B}\supset\bigcup_{i=1}^\infty\cl{A_i}$. Show, by an example, that this inclusion can be proper.
    \begin{solution}
      Let $p\in\cl{A_i}$ for any $i\in\N$. If $p\in A_i$ then $p\in B\subset\cl{B}$. Otherwise $p$ is a limit point of $A_i$, so it is a limit point of $B$, hence $p\in\cl{B}$. It follows that $\cl{B}\supset\bigcup_{i=1}^\infty\cl{A_i}$. Moreover this inclusion can be proper. Let $A_i=\{1/i\}$, for each $i\in\N$, and define $B=\bigcup_{i=1}^\infty A_i$. Then $\cl{A_i}=\{1/i\}$, so $B=\bigcup_{i=1}^\infty\cl{A_i}$. However $\cl{B}=B\cup\{0\}$, so $\bigcup_{i=1}^\infty\cl{A_i}$ is a proper subset of $\cl{B}$ in this case.
    \end{solution}
  \end{parts}

  \question Is every point of every open set $E\subset\R^2$ a limit point of $E$? Answer the same question for closed sets in $\R^2$.
  \begin{solution}
    Let $E\subset\R^2$ be open, and let $p\in E$. Then there is a neighbourhood $N_r(p)\subset E$. Let $\varepsilon>0$, and define $\delta=\min\{\varepsilon,r\}$. Then let $q=p+(\delta/2,0)$. Then $q\in N_\delta(p)$, so that $q\in N_\varepsilon(p)$ and $q\in N_r(p)\subset E$. Also, $q\neq p$. So all points $p$ of an open subset $E$ of $\R^2$ are limit points of $E$.

    The same does not hold for closed subsets of $\R^2$. For example the set $\{\vec{0}\}$ has no limit points, so is closed, and $\vec{0}$ is not a limit point of this set.
  \end{solution}

  \question Let $\interior{E}$ denote the set of all interior points of a set $E$. [See Definition 2.18(e); $\interior{E}$ is called the \emph{interior} of $E$.]
  \begin{parts}
    \part Prove that $\interior{E}$ is always open.
    \begin{solution}
      Let $p\in\interior{E}$, then there is a neighbourhood with $N_r(p)\subset E$. We claim that $N_r(p)\subset\interior{E}$. Let $q\in N_r(p)$, then since neighbourhoods are open there exists an $\varepsilon>0$ such that $N_\varepsilon(q)\subset N_r(p)\subset E$. Hence $q\in\interior{E}$, so that $N_r(p)\subset\interior{E}$. It follows that $\interior{E}$ is open.
    \end{solution}

    \part Prove that $E$ is open if and only if $\interior{E}=E$.
    \begin{solution}
      If $E$ is open then all points of $E$ are interior, hence $\interior{E}=E$. Conversely, if $\interior{E}=E$, then since $\interior{E}$ is always open so too is $E$.
    \end{solution}

    \part If $G\subset E$ and $G$ is open, prove that $G\subset\interior{E}$.
    \begin{solution}
      Let $g\in G$, then as $G$ is open it contains a neighbourhood $N_r(g)$. Hence $N_r(g)\subset G\subset E$, so that $g$ is interior to $E$. Hence $G\subset\interior{E}$.
    \end{solution}

    \part Prove that the complement of $\interior{E}$ is the closure of the complement of $E$.
    \begin{solution}
      Let $x\in\comp{(\interior{E})}$, then as $x$ is not interior to $E$, no neighbourhood of $x$ is completely contained in $E$. Hence every neighbourhood $N_r(x)$ contains a point in $\comp{E}$. If $x\notin\comp{E}$, then the points of $N_r(x)$ which are in $\comp{E}$ must be distinct from $x$, hence $x$ is a limit point of $\comp{E}$. Therefore either $x\in\comp{E}$ or $x\in(\comp{E})'$, and hence $\comp{(\interior{E})}\subset\cl{\comp{E}}$.

      Conversely, let $x\in\cl{\comp{E}}$. If $x\in\comp{E}$, then as $\interior{E}\subset E$ it follows that $x\in\comp{(\interior{E})}$. Otherwise, $x$ is a limit point of $\comp{E}$. Then every neighbourhood of $x$ contains points of $\comp{E}$, hence no neighbourhoods of $x$ are contained in $E$. It follows that $x$ is not interior to $E$, so $x\in\comp{(\interior{E})}$. Therefore $\cl{\comp{E}}\subset\comp{(\interior{E})}$, so $\comp{(\interior{E})}=\cl{\comp{E}}$.
    \end{solution}

    \part Do $E$ and $\cl{E}$ always have the same interiors?
    \begin{solution}
      No: consider the euclidean space $\R$, and let $E=\Q$. Since the irrationals are dense in $\R$, it follows that $\interior{E}=\emptyset$. However since the rationals are dense in $\R$, every real number is a limit point of $\Q$, hence $\cl{E}=\R$. Since $\R$ is open, it is its own interior. So $E$ and $\cl{E}$ do not have the same interior in this case.
    \end{solution}

    \part Do $E$ and $\interior{E}$ always have the same closures?
    \begin{solution}
      No: consider once again the euclidean space $\R$ and let $E=\Q$. We have seen that $\cl{E}=\R$, and that $\interior{E}=\emptyset$. Hence the closure of $\interior{E}$ is also empty, so $E$ and $\interior{E}$ do not have the same closures.
    \end{solution}
  \end{parts}

  \question Let $X$ be an infinite set. For $p\in X$ and $q\in X$, define
  \[ d(p,q) =
    \begin{cases}
      1 & \text{(if $p\neq q$)} \\
      0 & \text{(if $p=q$).}
    \end{cases}
  \]
  Prove that this is a metric. Which subsets of the resulting metric space are open? Which are closed? Which are compact?
  \begin{solution}
    It is clear that $d(p,q)=1>0$ when $p\neq q$, $d(p,p)=0$, and that $d(p,q)=d(q,p)$ in all cases. We need to prove the triangle inequality. Let $p,q,r\in X$. If $p\neq r$ and $q\neq r$, then
    \[ d(p,q) \leq 1 < 2 = d(p,r) + d(r,q). \]
    Otherwise without loss of generality $p=r$. Then $d(p,q)=d(r,q)=d(p,r)+d(r,q)$. So the triangle inequality holds in all cases, hence $d$ is a metric.

    All subsets of $X$ are open, for let $E\subset X$. If $p\in E$ then $N_{1/2}(x)=\{x\}\subset E$, hence all points of $E$ are interior, so $E$ is open. Since the complement of a set is open iff it is closed, it follows that all subsets of $X$ are also closed.

    We claim that the compact subsets of $X$ are precisely the finite subsets of $X$. First note any open cover $\{G_\alpha\}$ of a finite subset $K\subset X$ has a finite subcover. Each $k\in K$ is contained in some $G_{\alpha_k}$, and the finite subcollection $\{G_{\alpha_k}\}_{k\in K}$ covers $K$.

    Note that all subsets of $X$ are open, in particular singleton sets are open. Let $K\subset X$ be compact, then consider the open cover $\{\{k\}\}_{k\in K}$. Every element of $K$ is contained in precisely one element of the open cover, hence there are no proper subcovers. It follows that this open cover must be finite, hence $K$ is finite. So the compact subsets of $X$ are indeed the finite subsets of $X$.
  \end{solution}

  \question For $x\in\R^1$ and $y\in\R^1$, define
  \begin{align*}
    d_1(x,y) &= (x-y)^2, \\
    d_2(x,y) &= \sqrt{\abs{x-y}}, \\
    d_3(x,y) &= \abs{x^2-y^2}, \\
    d_4(x,y) &= \abs{x-2y}, \\
    d_5(x,y) &= \frac{\abs{x-y}}{1+\abs{x-y}}.
  \end{align*}
  Determine, for each of these, whether it is a metric or not.
  \begin{solution}
    The triangle inequality does not hold for $d_1$, as $d_1(0,1)+d(1,2)=1^2+1^2<2^2=d_1(0,2)$, hence $d_1$ is not a metric. We claim $d_2$ is a metric. It is clear that $d_2(x,y)>0$ when $x\neq y$, and that $d_2(x,x)=0$. Also $d_2(x,y)=d_2(y,x)$ holds for all $x,y$, since $\abs{x-y}=\abs{y-x}$. Let $x,y,z\in\R^1$, then
    \begin{align*}
      \abs{x-y} + \abs{y-z} &\geq \abs{x-z} \tag{Triangle Inequality} \\
      \implies \abs{x-y} + 2\sqrt{\abs{x-y}}\sqrt{\abs{y-z}} + \abs{y-z} &\geq \abs{x-z} \\
      \implies (\sqrt{\abs{x-y}} + \sqrt{\abs{y-z}})^2 &\geq \abs{x-z} \\
      \implies \sqrt{\abs{x-y}} + \sqrt{\abs{y-z}} &\geq \sqrt{\abs{x-z}},
    \end{align*}
    hence $d_2(x,y)+d_2(y,z)\geq d_2(x,z)$. So $d_2$ is a metric. Now $d_3$ and $d_4$ are not metrics, as $d_3(1,-1)=d_4(2,1)=0$, but $1\neq-1$ and $2\neq1$. We claim that $d_5$ is a metric. It is clear $d_5(x,y)>0$ when $x\neq y$, and that $d_5(x,x)=0$. Also $d_5(x,y)=d_5(y,x)$. Let $x,y,z\in\R^1$, then
    \begin{align*}
      \frac{\abs{x-y}}{1+\abs{x-y}} + \frac{\abs{y-z}}{1+\abs{y-z}} &\geq \frac{\abs{x-y}}{1+\abs{x-y}+\abs{y-z}} + \frac{\abs{y-z}}{1+\abs{x-y}+\abs{y-z}} \\
                                                                    &= 1 - \frac{1}{1+\abs{x-y}+\abs{y-z}} \\
                                                                    &\geq 1 - \frac{1}{1+\abs{x-z}} \tag{Triangle Inequality} \\
                                                                    &= \frac{\abs{x-z}}{1+\abs{x-z}},
    \end{align*}
    hence $d_5(x,y)+d_5(y,z)\geq d_5(x,z)$. So $d_5$ is a metric.
  \end{solution}

  \question Let $K\subset\R^1$ consist of 0 and the numbers $1/n$, for $n=1,2,3,$ \ldots. Prove that $K$ is compact directly from the definition (without using the Heine-Borel theorem).
  \begin{solution}
    Let $\{G_\alpha\}$ be an open over of $K$. There must be an open set $G_{\alpha_0}$ in the cover containing 0. Since $G_{\alpha_0}$ is open, there is a neighbourhood $N_r(0)\subset G_{\alpha_0}$, so that $1/n\in G_{\alpha_0}$ whenever $1/n<r$. Let $N\in\N$ be such that $Nr>1$. For each $n=1,\ldots,N-1$, there exists an open set $G_{\alpha_n}$ in the cover such that $1/n\in G_{\alpha_n}$.

    We claim that the finite subcollection $\{G_{\alpha_0}, G_{\alpha_1}, \ldots, G_{\alpha_{N-1}}\}$ covers $K$. Note $0\in G_{\alpha_0}$, and $1/n\in G_{\alpha_0}$ whenever $nr>1$. If $nr\leq 1$, then $n\in\{1,\ldots,N-1\}$ and so $n\in G_{\alpha_n}$. So every open cover of $K$ has a finite subcover, hence $K$ is compact.
  \end{solution}

  \question Construct a compact set of real numbers whose limit points for a countable set.

  \question Give an example of an open cover of the segment $(0,1)$ which has no finite subcover.
  \begin{solution}
    Define $G_n=(1/n,1)$, for $n=2,3,4,\ldots.$ Then $\{G_n\}$ is an open cover of $(0,1)$, since for every $x\in(0,1)$ there exists an $n\in\N$, $n\geq2$ with $nx>1$, i.e. $1/n<x<1$, so $x\in G_n$.

    Let $\{G_{n_1},\ldots,G_{n_k}\}$ be a finite subcollection of $\{G_n\}$, and take $N=\max\{n_1,\ldots,n_k\}$. Then $G_{n_j}\subset G_N=(1/N,1)$ for each $j=1,\ldots,k$, and $(1/N,1)\subsetneq(0,1)$, hence $\{G_{n_1},\ldots,G_{n_k}\}$ does not cover $(0,1)$. Therefore $\{G_n\}$ has no finite subcover.
  \end{solution}

  \question Show that Theorem 2.36 and its Corollary become false (in $\R^1$, for example) if the word ``compact'' is replaced by ``closed'' or by ``bounded.''
  \begin{solution}
    Theorem 2.36 is if $\{K_\alpha\}$ is a collection of compact subsets of a metric space $X$ such that the intersection of every finite subcollection of $\{K_\alpha\}$ is nonempty, then $\bigcap K_\alpha$ is nonempty. Its Corollary is if $\{K_n\}$ is a sequence of nonempty compact sets such that $K_n\supset K_{n+1}$ ($n=1,2,3,\ldots$), then $\bigcap_1^\infty K_n$ is nonempty.

    The Corollary is simply a special case of the Theorem, so it suffices to find counterexamples to the Corollary with the word ``compact'' replaced by ``closed'', and replaced by ``bounded''.

    Define the sequence $\{F_n\}$ of closed sets by $F_n=[n,\infty)$. Then each $F_n$ is nonempty and $F_n\supset F_{n+1}$, however $\bigcap_1^\infty F_n=\emptyset$, as for each $x\in\R$ there exists an $n\in\N$ with $n>x$, and then $x\notin F_n$.

    Define the sequence $\{G_n\}$ of bounded sets by $G_n=(0,1/n]$. Then each $G_n$ is nonempty and $G_n\supset G_{n+1}$, however $\bigcap_1^\infty G_n=\emptyset$, because $x\in G_n$ iff $1/x\in F_n$, and we already saw $\bigcap_1^\infty F_n=\emptyset$.
  \end{solution}

  \question Regard $\Q$, the set of all rational numbers, as a metric space, with $d(p,q)=\abs{p-q}$. Let $E$ be the set of all $p\in\Q$ such that $2<p^2<3$. Show that $E$ is closed and bounded in $\Q$, but that $E$ is not compact. Is $E$ open in $\Q$?
  \begin{solution}
    Define $F\subset\R$ and $G\subset\R$ by
    \[ F = [-\sqrt{3},-\sqrt{2}] \cup [\sqrt{2},\sqrt{3}], \qquad G = (-\sqrt{3},-\sqrt{2}) \cup (\sqrt{2},\sqrt{3}). \]
    Clearly $F$ is a closed subset of $\R$, and $G$ is an open subset of $\R$. Also
    \[ 2 < p^2 < 3 \iff -\sqrt{3} < p < -\sqrt{2} \quad \text{or} \quad \sqrt{2} < p < \sqrt{3}. \]
    Since no rational $p$ satisfies $p^2=2$ or $p^2=3$, it follows that $E=F\cap\Q=G\cap\Q$. Hence $E$ is a closed and open subset of $\Q$. Also if $d(0,p)\geq2$, then $p^2\geq4$, hence $p\notin E$. Therefore $d(0,p)<2$ for all $p\in E$ and so $E$ is bounded.

    Define $G_n=(-\infty,\sqrt{3}-1/n)\subset\R$ for each $n\in\N$, then $G_n\cap\Q$ is open in $\Q$ for all $n$. We claim $\{G_n\cap\Q\}$ is an open cover of $E$. If $p\in E$ then $p\in\Q$ and $p<\sqrt{3}$. Let $n\in\N$ be such that $n(\sqrt{3}-p)>1$. Then $p<\sqrt{3}-1/n$, so that $p\in G_n\cap\Q$. Therefore $E\subset\bigcup_1^\infty (G_n\cap\Q)$.

    Let $\{G_{n_1},\ldots,G_{n_k}\}$ be a finite subcollection of $\{G_n\}$. Take $N=\max\{n_1,\ldots,n_k\}$, then $G_{n_j}\subset G_N$ for all $j=1,\ldots,k$. Due to the density of $\Q$ in $\R$, there is a $p\in\Q$ satisfying $\max\{\sqrt{2},\sqrt{3}-1/N\}<p<\sqrt{3}$. Hence $p\in E$, however $p\notin G_N$ so that $E\not\subset G_N\cap\Q$. Therefore $\{G_{n_1}\cap\Q,\ldots,G_{n_k}\cap\Q\}$ does not cover $E$, hence the cover $\{G_n\cap\Q\}$ of $E$ has no finite subcovers. It follows that $E$ is not compact.    
  \end{solution}

  \question Let $E$ be the set of all $x\in[0,1]$ whose decimal expansion contains only the digits 4 and 7. Is $E$ countable? Is $E$ dense in $[0,1]$? Is $E$ compact? Is $E$ perfect?

  \question Is there a nonempty perfect set in $\R^1$ which contains no rational number?
  
  \question
  \begin{parts}
    \part If $A$ and $B$ are disjoint closed sets in some metric space $X$, prove that they are separated.

    \part Prove the same for disjoint open sets.

    \part Fix $p\in X$, $\delta>0$, define $A$ to be the set of all $q\in X$ for which $d(p,q)<\delta$, define $B$ similarly, with $>$ in place of $<$. Prove that $A$ and $B$ are separated.

    \part Prove that every connected metric space with at least two points is uncountable. \emph{Hint:} Use (c).
  \end{parts}

  \question Are closures and interiors of connected sets always connected? (Look at subset of $\R^2$.)

  \question Let $A$ and $B$ be separated subsets of some $\R^k$, suppose $\vec{a}\in A$, $\vec{b}\in B$, and define
  \[ \vec{p}(t) = (1-t)\vec{a} + t\vec{b} \]
  for $t\in\R^1$. Put $A_0=\vec{p}^{-1}(A)$, $B_0=\vec{p}^{-1}(B)$. [Thus $t\in A_0$ if and only if $\vec{p}(t)\in A$.]
  \begin{parts}
    \part Prove that $A_0$ and $B_0$ are separated subsets of $\R^1$.

    \part Prove that there exists $t_0\in(0,1)$ such that $\vec{p}(t_0)\notin A\cup B$.

    \part Prove that every convex subset of $\R^k$ is connected.
  \end{parts}

  \question A metric space is called separable if it contains a countable dense subset. Show that $\R^k$ is separable. \emph{Hint:} Consider the set of points which have only rational coordinates.

  \question A collection $\{V_\alpha\}$ of open subsets of $X$ is said to be a \emph{base} for $X$ if the following is true: For every $x\in X$ and every open set $G\subset X$ such that $x\in G$, we have $x\in V_\alpha\subset G$ for some $\alpha$. In other words, every open set in $X$ is the union of a subcollection of $\{V_\alpha\}$.

  Prove that every separable metric space has a \emph{countable} base. \emph{Hint:} Take all neighbourhoods with rational radius and center in some countable dense subset of $X$.

  \question Let $X$ be a metric space in which every infinite subset has a limit point. Prove that $X$ is separable. \emph{Hint:} Fix $\delta>0$, and pick $x_1\in X$. Having chosen $x_1,\ldots,x_j\in X$, choose $x_{j+1}\in X$, if possible, so that $d(x_1,x_{j+1})\geq\delta$ for $i=1,\ldots,j$. Show that this process must stop after a finite number of steps, and that $X$ can therefore be covered by finitely many neighbourhoods of radius $\delta$. Take $\delta=1/n$ ($n=1,2,3,\ldots$), and consider the centers of the corresponding neighbourhoods.

  \question Prove that every compact metric space $K$ has a countable base, and that $K$ is therefore separable. \emph{Hint:} For every positive integer $n$ there are finitely many neighbourhoods of radius $1/n$ whose union covers $K$.

  \question Let $X$ be a metric space in which every infinite subset has a limit point. Prove that $X$ is compact. \emph{Hint:} By Exercises 23 and 24, $X$ has a countable base. It follows that every open cover of $X$ has a \emph{countable} subcover $\{G_n\}$, $n=1,2,3,$ \ldots. If no finite subcollection of $\{G_n\}$ covers $X$, then the complement $F_n$ of $G_1\cup\cdots\cup G_n$ is nonempty for each $n$, but $\bigcap F_n$ is empty. If $E$ is a set which contains a point from each $F_n$, consider a limit point of $E$, and obtain a contradiction.

  \question Define a point $p$ in a metric space $X$ to be a \emph{condensation point} of a set $E\subset X$ if every neighbourhood of $p$ contains uncountably many points of $E$.

  Suppose $E\subset\R^k$, $E$ is uncountable, and let $P$ be the set of all condensation points of $E$. Prove that $P$ is perfect and that at most countably many points of $E$ are not in $P$. In other words, show that $P^c\cap E$ is at most countable. \emph{Hint:} Let $\{V_n\}$ be a countable base of $\R^k$, let $W$ be the union of those $V_n$ for which $E\cap V_n$ is at most countable, and show that $P=W^c$.

  \question Prove that every closed set in a separable metric space is the union of a (possibly empty) perfect set and a set which is at most countable (\emph{Corollary:} Every countable closed set in $\R^k$ has isolated points.) \emph{Hint:} Use Exercise 27.

  \question Prove that every open set in $\R^1$ is the union of an at most countable collection of disjoint segments. \emph{Hint:} Use Exercise 22.

  \question Imitate the proof of Theorem 2.43 to obtain the following result:

  If $\R^k=\bigcup_1^\infty F_n$, where each $F_n$ is a closed subset of $\R^k$, then at least one $F_n$ has a nonempty interior.

  \emph{Equivalent statement:} If $G_n$ is a dense open subset of $\R^k$, for $n=1,2,3,$ \ldots, then $\bigcap_1^\infty G_n$ is not empty (in fact, it is dense in $\R^k$).

  (This is a special case of Baire's theorem; see Exercise 22, Chap. 3, for the general case.)
\end{questions}

%%% Local Variables:
%%% mode: latex
%%% TeX-master: "rudin"
%%% End:
