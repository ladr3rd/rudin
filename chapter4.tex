\section{Continuity}

\begin{questions}
  \question Suppose $f$ is a real function defined on $\R^1$ which satisfies
  \[ \lim_{h\to0} [f(x+h) - f(x-h)] = 0 \]
  for every $x\in\R^1$. Does imply that $f$ is continuous?

  \question If $f$ is a continuous mapping of a metric space $X$ into a metric space $Y$, prove that
  \[ f(\cl{E})\subset\cl{f(E)} \]
  for every set $E\subset X$. ($\cl{E}$ denotes the closure of $E$.) Show, by an example, that $f(\cl{E})$ can be a proper subset of $\cl{f(E)}$.

  \question Let $f$ be a continuous real function on a metric space $X$. Let $Z(f)$ (the \emph{zero set} of $f$) be the set of all $p\in X$ at which $f(p)=0$. Prove that $Z(f)$ is closed.

  \question Let $f$ and $g$ be continuous mappings of a metric space $X$ into a metric space $Y$, and let $E$ be a dense subset of $X$. Prove that $f(E)$ is dense in $f(X)$. If $g(p)=f(p)$ for all $p\in E$, prove that $g(p)=f(p)$ for all $p\in X$. (In other words, a continuous mapping is determined by its values on a dense subset of its domain.)

  \question If $f$ is a real continuous function defined on closed set $E\subset\R^1$, prove that there exist continuous real functions $g$ on $\R^1$ such that $g(x)=f(x)$ for all $x\in E$. (Such functions $g$ are called \emph{continuous extensions} of $f$ from $E$ to $\R^1$.) Show that the result becomes false if the word ``closed'' is omitted. Extend the result to vector-valued functions. \emph{Hint:} Let the graph of $g$ be a straight line on each of the segments which constitute the complement of $E$ (compare Exercise 29, Chap. 2). The result remains true if $\R^1$ is replaced by any metric space, but the proof is not so simple.

  \question If $f$ is defined on $E$, the \emph{graph} of $f$ is the set of points $(x,f(x))$, for $x\in E$. In particular, if $E$ is a set of real numbers, and $f$ is real-valued, the graph of $f$ is a subset of the plane.

  Suppose $E$ is compact, and prove that $f$ is continuous on $E$ if and only if its graph is compact.

  \question If $E\subset X$ and if $f$ is a function defined on $X$, the \emph{restriction} of $f$ to $E$ is the function $g$ whose domain of definition is $E$, such that $g(p)=f(p)$ for $p\in E$. Define $f$ and $g$ on $\R^2$ by: $f(0,0)=g(0,0)=0$, $f(x,y)=xy^2/(x^2+y^4)$, $g(x,y)=xy^2/(x^2+y^6)$ if $(x,y)\neq(0,0)$. Prove that $f$ is bounded on $\R^2$, that $g$ is unbounded in every neighbourhood of $(0,0)$, and that $f$ is not continuous at $(0,0)$; nevertheless, the restrictions of both $f$ and $g$ to every straight line in $\R^2$ are continuous!

  \question Let $f$ be a real uniformly continuous function on the bounded set $E$ in $\R^1$. Prove that $f$ is bounded on $E$.

  Show that the conclusion is false if boundedness of $E$ is omitted from the hypothesis.

  \question Show that the requirement in the definition of uniform continuity can be rephrased as follows, in terms of diameters of sets: To every $\varepsilon>0$ there exists a $\delta>0$ such that $\diam f(E)<\varepsilon$ for all $E\subset X$ with $\diam E<\delta$.

  \question Complete the details of the following alternative proof of Theorem 4.19: If $f$ is not uniformly continuous, then for some $\varepsilon>0$ there are sequences $\{p_n\}$, $\{q_n\}$ in $X$ such that $d_X(p_n,q_n)\to0$ but $d_Y(f(p_n),f(q_n))>\varepsilon$. Use Theorem 2.37 to obtain a contradiction.

  \question Suppose $f$ is a uniformly continuous mapping of a metric space $X$ into a metric space $Y$ and prove that $\{f(x_n)\}$ is a Cauchy sequence in $Y$ for every Cauchy sequence $\{x_n\}$ in $X$. Use this result to give an alternative proof of the theorem stated in Exercise 13.

  \question A uniformly continuous function of a uniformly continuous function is uniformly continuous.

  State this more precisely and prove it.

  \question Let $E$ be a dense subset of a metric space $X$, and let $f$ be a uniformly continuous \emph{real} function defined on $E$. Prove that $f$ has a continuous extension from $E$ to $X$ (see Exercise 5 for terminology). (Uniqueness follows from Exercise 4.) \emph{Hint:} For each $p\in X$ and each positive integer $n$, let $V_n(p)$ be the set of all $q\in E$ with $d(p,q)<1/n$. Use Exercise 9 to show that the intersection of the closures of the sets $f(V_1(p)),f(V_2(p)),\ldots,$ consists of a single point, say $g(p)$, of $\R^1$. Prove that the function $g$ so defined on $X$ is the desired extension of $f$.

  Could the range space $\R^1$ be replaced by $\R^k$? By any compact metric space? By any complete metric space? By any metric space?

  \question Let $I=[0,1]$ be the closed unit interval. Suppose $f$ is a continuous mapping of $I$ into $I$. Prove that $f(x)=x$ for at least one $x\in I$.

  \question Call a mapping of $X$ into $Y$ \emph{open} if $f(V)$ is an open set in $Y$ whenever $V$ is an open set in $X$.

  Prove that every continuous open mapping of $\R^1$ into $\R^1$ is monotonic.

  \question Let $[x]$ denote the largest integer contained in $x$, that is, $[x]$ is the integer such that $x-1<[x]\leq x$; and let $(x)=x-[x]$ denote the fractional part of $x$. What discontinuities do the functions $[x]$ and $(x)$ have?

  \question Let $f$ be a real function defined on $(a,b)$. Prove that the set of points at which $f$ has a simple discontinuity is at most countable. \emph{Hint:} Let $E$ be the set on which $f(x-)<f(x+)$. With each point $x$ of $E$, associate a triple $(p,q,r)$ of rational numbers such that
  \begin{enumerate}[label=(\alph*)]
  \item $f(x-)<p<f(x+)$,
  \item $a<q<t<x$ implies $f(t)<p$,
  \item $x<t<r<b$ implies $f(t)>p$.
  \end{enumerate}
  The set of all such triples is countable. Show that each triple is associated with at most one point of $E$. Deal similarly with the other possible types of simple discontinuities.

  \question Every rational $x$ can be written in the form $x=m/n$, where $n>0$, and $m$ and $n$ are integers without any common divisors. When $x=0$, we take $n=1$. Consider the function $f$ defined on $\R^1$ by
  \[ f(x) =
    \begin{cases}
      0 & \text{($x$ irrational),} \\[0.5em]
      \dfrac{1}{n} & \left( x = \dfrac{m}{n} \right).
    \end{cases}
  \]
  Prove that $f$ is continuous at every irrational point, and that $f$ has a simple discontinuity at every rational point.

  \question Suppose $f$ is a real function with domain $\R^1$ which has the intermediate value property: If $f(a)<c<f(b)$, then $f(x)=c$ for some $x$ between $a$ and $b$.

  Suppose also, for every rational $r$, that the set of all $x$ with $f(x)=r$ is closed.

  Prove that $f$ is continuous.

  \emph{Hint:} If $x_n\to x_0$ but $f(x_n)>r>f(x_0)$ for some $r$ and all $n$, then $f(t_n)=r$ for some $t_n$ between $x_0$ and $x_n$; thus $t_n\to x_0$. Find a contradiction. (N. J. Fine, Amer. Math. Monthly, vol. 73, 1966, p. 782.)
  \begin{solution}
    Suppose that $f$ is not continuous at $x_0$. Consider a sequence $y_n\to x_0$ with $f(y_n)\not\to f(x_0)$. Without loss of generality infinitely many $y_n$ satisfy $f(y_n)>f(x_0)$. Let $(y_{n_k})$ denote the subsequence of $(y_n)$ consisting of all such terms. If only finitely many of the $y_n$ satisfy $f(y_n)<f(x_0)$, then we must have $f(y_{n_k})\not\to f(x_0)$.

    Otherwise we may define $(y_{m_k})$ to be the subsequence of $(y_n)$ consisting of all terms with $f(y_n)<f(x_0)$. Suppose that $f(y_{n_k})\to f(x_0)$ and $f(y_{m_k})\to f(x_0)$. Then every neighborhood of $f(x_0)$ contains all but finitely many terms $f(y_{n_k})$, and all but finitely many terms $f(y_{m_k})$. Therefore every neighbourhood of $f(x_0)$ contains all but finitely many terms $f(y_n)$. This contradicts $f(y_n)\not\to f(x_0)$, hence one of $(f(y_{n_k}))$ and $(f(y_{m_k}))$ must not converge to $f(x_0)$.

    Without loss of generality there is a sequence $x_n\to x_0$ with $f(x_n)>f(x_0)$ for all $n$, and $f(x_n)\not\to f(x_0)$. So there is an $\varepsilon>0$ where $f(x_n)>f(x_0)+\varepsilon$ for all $n$. Let $r$ be rational with
    \[ f(x_n) > f(x_0) + \varepsilon > r > f(x_0). \]
    Then for each $n$, by the intermediate value property, $f(t_n)=r$ for some $t_n$ between $x_0$ and $x_n$; thus $t_n\to x_0$. So $x_0$ is a limit point of $f^{-1}(\{r\})$, and $f(x_0)\neq r$, which contradicts $f^{-1}(\{r\})$ being closed. This is a contradiction, hence $f$ is continuous.
  \end{solution}

  \question If $E$ is a nonempty subset of a metric space $X$, define the distance from $x\in X$ to $E$ by
  \[ \rho_E(x) = \inf_{x\in E} d(x,z). \]
  \begin{parts}
    \part Prove that $\rho_E(x)=0$ if and only if $x\in\cl{E}$.

    \part Prove that $\rho_E$ is a uniformly continuous function on $X$, by showing that
    \[ \abs{\rho_E(x) - \rho_E(y)} \leq d(x,y) \]
    for all $x\in X$, $y\in X$.

    \emph{Hint:} $\rho_E(x) \leq d(x,z) \leq d(x,y) + d(y,z)$, so that
    \[ \rho_E(x) \leq d(x,y) + \rho_E(y). \]
  \end{parts}

  \question Suppose $K$ and $F$ are disjoint sets in a metric space $X$, $K$ is compact, $F$ is closed. Prove that there exists $\delta>0$ such that $d(p,q)>\delta$ if $p\in K$, $q\in F$. \emph{Hint:} $\rho_F$ is a continuous positive function on $K$.

  Show that the conclusion may fail for two disjoint closed sets if neither is compact.

  \question Let $A$ and $B$ be disjoint nonempty closed sets in a metric space $X$, and define
  \[ f(p) = \frac{\rho_A(p)}{\rho_A(p) + \rho_B(p)} \quad (p\in X). \]
  Show that $f$ is a continuous function on $X$ whose range lies in $[0,1]$, that $f(p)=0$ precisely on $A$ and $f(p)=1$ precisely on $B$. This establishes a converse of Exercise 3: Every closed set $A\subset X$ is $Z(f)$ for some continuous real $f$ on $X$. Setting
  \[ V=f^{-1}([0,\tfrac{1}{2})), \qquad W = f^{-1}((\tfrac{1}{2},1]), \]
  show that $V$ and $W$ are open and disjoint, and that $A\subset V$, $B\subset W$. (Thus pairs of disjoint closed sets in a metric space can be covered by pairs of disjoint open sets. This property of metric spaces is called \emph{normality}.)

  \question A real-valued function $f$ defined in $(a,b)$ is said to be \emph{convex} if
  \[ f(\lambda x + (1-\lambda)y) \leq \lambda f(x) + (1-\lambda)f(y) \]
  whenever $a<x<b$, $a<y<b$, $0<\lambda<1$. Prove that every convex function is continuous. Prove that every increasing convex function of a convex function is convex. (For example, if $f$ is convex, so is $e^f$.)

  If $f$ is convex in $(a,b)$ and if $a<s<t<u<b$, show that
  \[ \frac{f(t)-f(s)}{t-s} \leq \frac{f(u)-f(s)}{u-s} \leq \frac{f(u)-f(t)}{u-t}. \]

  \question Assume that $f$ is a continuous real function defined in $(a,b)$ such that
  \[ f\left( \frac{x+y}{2} \right) \leq \frac{f(x)+f(y)}{2} \]
  for all $x,y\in(a,b)$. Prove that $f$ is convex.

  \question If $A\subset\R^k$ and $B\subset\R^k$, define $A+B$ to be the set of all sums $\vec{x}+\vec{y}$ with $\vec{x}\in A$ and $\vec{y}\in B$.
  \begin{parts}
    \part If $K$ is compact and $C$ is closed in $\R^k$, prove that $K+C$ is closed.

    \emph{Hint:} Take $\vec{z}\notin K+C$, put $F=\vec{z}-C$, the set of all $\vec{z}-\vec{y}$ with $\vec{y}\in C$. Then $K$ and $F$ are disjoint. Choose $\delta$ as in Exercise 21. Show that the open ball with center $\vec{z}$ and radius $\delta$ does not intersect $K+C$.

    \part Let $\alpha$ be an irrational real number. Let $C_1$ be the set of all integers, let $C_2$ be the set of $n\alpha$ with $n\in C_1$. Show that $C_1$ and $C_2$ are closed subsets of $\R^1$ whose sum $C_1+C_2$ is \emph{not} closed, by showing that $C_1+C_2$ is a countable dense subset of $\R^1$.
  \end{parts}

  \question Suppose $X$, $Y$, $Z$ are metric spaces, and $Y$ is compact. Let $f$ map $X$ into $Y$, let $g$ be a continuous one-to-one mapping of $Y$ into $Z$, and put $h(x)=g(f(x))$ for $x\in X$.

  Prove that $f$ is uniformly continuous if $h$ is uniformly continuous.

  \emph{Hint:} $g^{-1}$ has compact domain $g(Y)$, and $f(x)=g^{-1}(h(x))$.

  Prove also that $f$ is continuous if $h$ is continuous.

  Show (by modifying Example 4.21, or by finding a different example) that the compactness of $Y$ cannot be omitted from the hypothesis, even when $X$ and $Z$ are compact.
\end{questions}

%%% Local Variables:
%%% mode: latex
%%% TeX-master: "rudin"
%%% End:
